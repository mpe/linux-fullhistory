\documentclass{article}
\def\version{$Id: cdrom-standard.tex,v 0.4 1996/04/17 20:46:34 david Exp $}

\evensidemargin=0pt
\oddsidemargin=0pt
\topmargin=-\headheight \advance\topmargin by -\headsep
\textwidth=15.99cm \textheight=24.62cm % normal A4, 1'' margin

\def\linux{{\sc Linux}}
\def\cdrom{{\sc CDrom}}
\def\cdromc{{\tt cdrom.c}}
\def\ucdrom{{\tt ucdrom.h}}

\everymath{\it} \everydisplay{\it}
\catcode `\_=\active \def_{\_\penalty100 }
\catcode`\<=\active \def<#1>{{\langle\hbox{\rm#1}\rangle}}

\begin{document}
\title{A \linux\ \cdrom\ standard}
\author{David van Leeuwen\\{\normalsize\tt david@tm.tno.nl}}

\maketitle

\section{Introduction}

\linux\ is probably the Unix-like operating system that supports the widest
variety of hardware devices. The reasons for this are presumably
\begin{itemize}
\item The large list of different hardware devices available for the popular
IBM PC-architecture,
\item The open design of the operating system, such that everybody can
write a driver for Linux.
\end{itemize}
The vast choice and openness has lead not only to a wide support of
hardware devices, but also to a certain divergence in
behavior. Especially for \cdrom\ devices, the way a particular drive
reacts to a `standard' $ioctl()$ call varies a lot from one brand
to another. 

Undoubtedly, this has a reason. Since the beginning of the \cdrom,
many different interfaces developed. Most of them had proprietary
interfaces, which means that a separate driver had to be written for
each new type of interface. Nowadays, all new \cdrom\ types are either
ATAPI/IDE or SCSI. But history has delivered us \cdrom\ support for
some 10 or so different interfaces. Not all drives have the same
capabilities, and all different interfaces use different i/o formats
for the data. For the interfacing with the \linux\ operating system
and software, this has lead to a rather wild set of commands and data
formats. Presumably, every \cdrom\ device drive author has added his
own set of ioctl commands and used a format reminiscent of the
underlying hardware. Any structure is lost. 

Apart from the somewhat unstructured interfacing with software, the
actual execution of the commands is different for most of the
different drivers: e.g., some drivers close the tray if an $open()$ call
occurs while the tray is unloaded, others not. Some drivers lock the
door upon opening the device, to prevent an incoherent file system,
but others don't, to allow software ejection. Undoubtedly, the
capabilities of the different drives vary, but even when two drives have
the same capability the driver behavior may be different. 

Personally, I think that the most important drive interfaces will be
the IDE/ATAPI drives and of course the SCSI drives, but as prices of
hardware drop continuously, it is not unlikely that people will have
more than one \cdrom\ drive, possibly of mixed types.  (In December
1994, one of the cheapest \cdrom\ drives was a Philips cm206, a
double-speed proprietary drive. In the months that I was busy writing
a \linux\ driver for it, proprietary drives became old-fashioned and
IDE/ATAPI drives became standard. At the time of writing (April 1996)
the cheapest double speed drive is IDE and at one fifth of the price
of its predecessor. Eight speed drives are available now.)

This document defines (in pre-release versions: proposes) the various
$ioctl$s, and the way the drivers should implement this.

\section{Standardizing through another software level}
\label{cdrom.c}

At the time this document is written, all drivers directly implement
the $ioctl()$ calls through their own routines, with the danger of
forgetting calls to $verify_area()$ and the risk of divergence in
implementation. 

For this reason, we\footnote{The writing style is such that `we' is
used when (at least part of) the \cdrom-device driver authors support
the idea, an `I' is used for personal opinions} propose to define
another software-level, that separates the $ioctl()$ and $open()$
implementation from the actual hardware implementation. We believe
that \cdrom\ drives are specific enough (i.e., different from other
block-devices such as floppy or hard disc drives), to define a set of
{\em \cdrom\ device operations}, $<cdrom-device>_dops$. These are of a
different nature than the classical block-device file operations
$<block-device>_fops$.

The extra interfacing level routines are implemented in a file
\cdromc, and a low-level \cdrom\ driver hands over the interfacing to
the kernel by registering the following general $struct\ file_operations$:
$$
\halign{$#$\ \hfil&$#$\ \hfil&$/*$ \rm# $*/$\hfil\cr
struct& file_operations\ cdrom_fops = \{\hidewidth\cr
        &NULL,                  & lseek \cr
        &block_read,            & read - general\ block-dev\ read \cr
        &block_write,           & write - general block-dev write \cr
        &NULL,                  & readdir \cr
        &NULL,                  & select \cr
        &cdrom_ioctl,           & ioctl \cr
        &NULL,                  & mmap \cr
        &cdrom_open,            & open \cr
        &cdrom_release,         & release \cr
        &NULL,                  & fsync \cr
        &NULL,                  & fasync \cr
        &cdrom_media_changed,   & media_change \cr
        &NULL                   & revalidate \cr
\};\cr
}
$$
Every active \cdrom\ device shares this $struct$. The routines declared
above are all implemented in \cdromc, and this is the place where the
{\em behavior\/} of all \cdrom-devices is defined, and hence
standardized. The implementation of the interfacing to the various
types of hardware still is done by the various \cdrom-device drivers,
but these routines only implement certain {\em capabilities\/} that
are typical to \cdrom\ (removable-media) devices.

Registration of the \cdrom\ device driver should now be to the general
routines in \cdromc, not to the VFS any more. This is done though the
call
$$register_cdrom(int\ major, char * name, 
  struct\ cdrom_device_ops\ device_options)
$$

The device operations structure lists the implemented routines for
interfacing to the hardware, and some specifications of capabilities
of the device, such as the maximum head-transfer rate.  [It is
impossible to come up with a complete list of all capabilities of
(future) \cdrom\ drives, as the developments in technology follow-up
at an incredible rate. Maybe write-operation (WORM devices) will
become very popular in the future.]  The list now is:
$$
\halign{$#$\ \hfil&$#$\ \hfil&\hbox to 10em{$#$\hss}&
  $/*$ \rm# $*/$\hfil\cr
struct& cdrom_device_ops\ \{ \hidewidth\cr
  &int& (* open)(dev_t, int)\cr
  &void& (* release)(dev_t);\cr 
  &int& (* open_files)(dev_t);  \cr
  &int& (* drive_status)(dev_t);\cr     
  &int& (* disc_status)(dev_t);\cr      
  &int& (* media_changed)(dev_t);\cr 
  &int& (* tray_move)(dev_t, int);\cr
  &int& (* lock_door)(dev_t, int);\cr
  &int& (* select_speed)(dev_t, int);\cr
  &int& (* select_disc)(dev_t, int);\cr
  &int& (* get_last_session) (dev_t, struct\ cdrom_multisession *{});\cr
  &int& (* get_mcn)(dev_t, struct\ cdrom_mcn *{});\cr
  &int& (* reset)(dev_t);\cr
  &int& (* audio_ioctl)(dev_t, unsigned\ int, void *{});\cr 
  &int& (* dev_ioctl)(dev_t, unsigned\ int, unsigned\ long);\cr
\noalign{\medskip}
  &\llap{const\ }int& capability;&  capability flags \cr
  &int& mask;& mask of capability: disables them \cr
  &\llap{$const\ $}float& speed;&  maximum speed for reading data \cr
  &\llap{$const\ $}int& minors;& number of supported minor devices \cr
  &\llap{$const\ $}int& capacity;& number of discs in jukebox \cr
\noalign{\medskip}
  &int& options;& options flags \cr
  &long& mc_flags;& media-change buffer flags ($2\times16$) \cr
\}\cr
}
$$ The \cdrom-driver should simply implement (some of) these
functions, and register the functions to the global \cdrom\ driver,
which performs interfacing with the Virtual File System and system
$ioctl$s. The flags $capability$ specify the hardware-capabilities on
registration of the device, the flags $mask$ can be used to mask some
of those capabilities (for one reason or another). The value $minors$
should be a positive value indicating the number of minor devices that
are supported by the driver, normally~1.  (They are supposed to be
numbered from 0 upwards). The value $capacity$ should be the number of
discs the drive can hold simultaneously, if it is designed as a
juke-box, or otherwise~1.

Two registers contain variables local to the \cdrom\ device. The flags
$options$ are used to specify how the general \cdrom\ routines
should behave. These various flags registers should provide enough
flexibility to adapt to the different user's wishes (and {\em not\/}
the `arbitrary' wishes of the author of the low-level device driver,
as is the case in the old scheme). The register $mc_flags$ is used to
buffer the information from $media_changed()$ to two separate queues. 

Note that most functions have fewer parameters than their
$blkdev_fops$ counterparts. This is because very little of the
information in the structures $inode$ and $file$ are used, the main
parameter is the device $dev$, from which the minor-number can be
extracted. (Most low-level \cdrom\ drivers don't even look at that value
as only one device is supported.)

The intermediate software layer that \cdromc\ forms will performs some
additional bookkeeping. The minor number of the device is checked
against the maximum registered in $<device>_dops$. The function
$cdrom_ioctl()$ will verify the appropriate user-memory regions for
read and write, and in case a location on the CD is transferred, it
will `sanitize' the format by making requests to the low-level drivers
in a standard format, and translating all formats between the
user-software and low level drivers. This relieves much of the drivers
memory checking and format checking and translation. Also, the
necessary structures will be declared on the program stack.

The implementation of the functions should be as defined in the
following sections. Three functions {\em must\/} be implemented,
namely $open()$, $release()$ and $open_files()$. Other functions may
be omitted, their corresponding capability flags will be cleared upon
registration. Generally, a function returns zero on success and
negative on error. A function call should return only after the
command has completed, but of course waiting for the device should not
use processor time.

\subsection{$Open(dev_t\ dev, int\ purpose)$}

$Open()$ should try to open the device for a specific $purpose$, which
can be either:
\begin{itemize}
\item[0] Open for data read, as is used by {\tt mount()} (2), or the
user commands {\tt dd} or {\tt cat}.  
\item[1] Open for $ioctl$ commanding, as is used for audio-CD playing
programs mostly. 
\end{itemize}
In this routine, a static counter should be updated, reflecting the
number of times the specific device is successfully opened (and in
case the driver supports modules, the call $MOD_INC_USE_COUNT$
should be performed exactly once, if successful). The return value is
negative on error, and zero on success. The open-for-ioctl call can
only fail if there is no hardware.

Notice that any strategic code (closing tray upon $open()$, etc.)\ is
done by the calling routine in \cdromc, so the low-level routine
should only be concerned with proper initialization and device-use
count.

\subsection{$Release(dev_t\ dev)$}

The use-count of the device $dev$ should be decreased by 1, and a
single call $MOD_DEC_USE_COUNT$ should be coded here.  Possibly other
device-specific actions should be taken such as spinning down the
device. However, strategic actions such as ejection of the tray, or
unlocking the door, should be left over to the general routine
$cdrom_release()$. Also, the invalidation of the allocated buffers in
the VFS is taken care of by the routine in \cdromc.

\subsection{$Open_files(dev_t\ dev)$}

This function should return the internal variable use-count of the
device $dev$. The use-count is not implemented in the routines in
\cdromc\ itself, because there may be many minor devices connected to
a single low-level driver.

\subsection{$Drive_status(dev_t\ dev)$}
\label{drive status}

The function $drive_status$, if implemented, should provide
information of the status of the drive (not the status of the disc,
which may or may not be in the drive). In \ucdrom\ the possibilities
are listed: 
$$
\halign{$#$\ \hfil&$/*$ \rm# $*/$\hfil\cr
CDS_NO_INFO& no information available\cr
CDS_NO_DISC& no disc is inserted, tray is closed\cr
CDS_TRAY_OPEN& tray is opened\cr
CDS_DRIVE_NOT_READY& something is wrong, tray is moving?\cr
CDS_DISC_OK& a disc is loaded and everything is fine\cr
}
$$

\subsection{$Disc_status(dev_t\ dev)$}
\label{disc status}

As a complement to $drive_status()$, this functions can provide the
general \cdrom-routines with information about the current disc that is
inserted in the drive represented by $dev$. The history of development
of the CD's use as a carrier medium for various digital information
has lead to many different disc types, hence this function can return:
$$
\halign{$#$\ \hfil&$/*$ \rm# $*/$\hfil\cr
CDS_NO_INFO& no information available\cr
CDS_NO_DISC& no disc is inserted, or tray is opened\cr
CDS_AUDIO& Audio disc (2352 audio bytes/frame)\cr
CDS_DATA_1& data disc, mode 1 (2048 user bytes/frame)\cr
CDS_DATA_2& data disc, mode 2 (2336 user bytes/frame)\cr
CDS_XA_2_1& mixed data (XA), mode 2, form 1 (2048 user bytes)\cr
CDS_XA_2_2& mixed data (XA), mode 2, form 1 (2324  user bytes)\cr
}
$$
As far as I know, \cdrom's are always of type $CDS_DATA_1$. For
some information concerning frame layout of the various disc types, see
a recent version of {\tt cdrom.h}. 

\subsection{$Media_changed(dev\_t\ dev)$}

This function is very similar to the original function in $struct\
file_operations$. It returns 1 if the medium of the device $dev$ has
changed since the last call, and 0 otherwise. Note that by `re-routing'
this function through $cdrom_media_changed()$, we can implement
separate queues for the VFS and a new $ioctl()$ function that can
report device changes to software (e.g., an auto-mounting daemon). 

\subsection{$Tray_move(dev_t\ dev, int\ position)$}

This function, if implemented, should control the tray movement. (No
other function should control this.) The parameter $position$ controls
the desired direction of movement:
\begin{itemize}
\item[0] Close tray
\item[1] Open tray
\end{itemize}
This function returns 0 upon success, and a non-zero value upon
error. Note that if the tray is already in the desired position, no
action need be taken, and the return value should be 0. 

\subsection{$Lock_door(dev_t\ dev, int\ lock)$}

This function (and no other code) controls locking of the door, if the
drive allows this. The value of $lock$ controls the desired locking
state:
\begin{itemize}
\item[0] Unlock door, manual opening is allowed
\item[1] Lock door, tray cannot be ejected manually
\end{itemize}
Return values are as for $tray_move()$.

\subsection{$Select_speed(dev_t\ dev, int\ speed)$}

Although none of the drivers has implemented this function so far,
some drives are capable of head-speed selection, and hence this is a
capability that should be standardized through a function in the
device-operations structure. This function should select the speed at
which data is read or audio is played back. The special value `0'
means `auto-selection', i.e., maximum data-rate or real-time audio
rate. If the drive doesn't have this `auto-selection' capability, the
decision should be made on the current disc loaded and the return
value should be positive. A negative return value indicates an
error. (Although the audio-low-pass filters probably aren't designed
for it, more than real-time playback of audio might be used for
high-speed copying of audio tracks). Badly pressed \cdrom s may
benefit from less-than-maximum head rate.

\subsection{$Select_disc(dev_t\ dev, int\ number)$}

If the drive can store multiple discs (a juke-box), it is likely that
a disc selection can be made by software. This function should perform
disc selection. It should return the number of the selected disc on
success, a negative value on error. Currently, none of the \linux\ 
\cdrom\ drivers appear to support such functionality, but it defined
here for future purpose.

\subsection{$Get_last_session(dev_t\ dev, struct\ cdrom_multisession *
ms_info)$}

This function should implement the old corresponding $ioctl()$. For
device $dev$, the start of the last session of the current disc should
be returned in the pointer argument $ms_info$. Note that routines in \cdromc\ have sanitized this argument: its
requested format will {\em always\/} be of the type $CDROM_LBA$
(linear block addressing mode), whatever the calling software
requested. But sanitization goes even further: the low-level
implementation may return the requested information in $CDROM_MSF$
format if it wishes so (setting the $ms_info\rightarrow addr_format$
field appropriately, of course) and the routines in \cdromc\ will make
the transform if necessary. The return value is 0 upon success.

\subsection{$Get_mcn(dev_t\ dev, struct\ cdrom_mcn * mcn)$}

Some discs carry a `Media Catalog Number' (MCN), also called
`Universal Product Code' (UPC). This number should reflect the number that
is generally found in the bar-code on the product. Unfortunately, the
few discs that carry such a number on the disc don't even use the same
format. The return argument to this function is a pointer to a
pre-declared memory region of type $struct\ cdrom_mcn$. The MCN is
expected as a 13-character string, terminated by a null-character.

\subsection{$Reset(dev_t dev)$}

This call should implement hard-resetting the drive (although in
circumstances that a hard-reset is necessary, a drive may very well
not listen to commands anymore). Preferably, control is returned to the
caller only after the drive has finished resetting.

\subsection{$Audio_ioctl(dev_t\ dev, unsigned\ int\ cmd, void *
arg)$}

Some of the \cdrom-$ioctl$s defined in {\tt cdrom.h} can be
implemented by the routines described above, and hence the function
$cdrom_ioctl$ will use those. However, most $ioctl$s deal with
audio-control. We have decided to leave these accessed through a
single function, repeating the arguments $cmd$ and $arg$. Note that
the latter is of type $void*{}$, rather than $unsigned\ long\
int$. The routine $cdrom_ioctl()$ does do some useful things,
though. It sanitizes the address format type to $CDROM_MSF$ (Minutes,
Seconds, Frames) for all audio calls. It also verifies the memory
location of $arg$, and reserves stack-memory for the argument. This
makes implementation of the $audio_ioctl()$ much simpler than in the
old driver scheme. For an example you may look up the function
$cm206_audio_ioctl()$ in {\tt cm206.c} that should be updated with
this documentation. 

An unimplemented ioctl should return $-EINVAL$, but a harmless request
(e.g., $CDROMSTART$) may be ignored by returning 0 (success). Other
errors should be according to the standards, whatever they are. (We
may decide to sanitize the return value in $cdrom_ioctl()$, in order
to guarantee a uniform interface to the audio-player software.)

\subsection{$Dev_ioctl(dev_t\ dev, unsigned\ int\ cmd, unsigned\ long\
arg)$}

Some $ioctl$s seem to be specific to certain \cdrom\ drives. That is,
they are introduced to service some capabilities of certain drives. In
fact, there are 6 different $ioctl$s for reading data, either in some
particular kind of format, or audio data. Not many drives support
reading audio tracks as data, I believe this is because of protection
of copyrights of artists. Moreover, I think that if audio-tracks are
supported, it should be done through the VFS and not via $ioctl$s. A
problem here could be the fact that audio-frames are 2352 bytes long,
so either the audio-file-system should ask for 75264 bytes at once
(the least common multiple of 512 and 2352), or the drivers should
bend their backs to cope with this incoherence (to which I would
oppose, this code then should be standardized in \cdromc).

Because there are so many $ioctl$s that seem to be introduced to
satisfy certain drivers,\footnote{Is there software around that actually uses
these? I 'd be interested!} any `non-standard' $ioctl$s are routed through
the call $dev_ioctl()$. In principle, `private' $ioctl$s should be
numbered after the device's major number, and not the general \cdrom\
$ioctl$ number, {\tt 0x53}. Currently the non-supported $ioctl$s are:
{\it CDROMREADMODE1, CDROMREADMODE2, CDROMREADAUDIO, CDROMREADRAW,
CDROMREADCOOKED, CDROMSEEK, CDROMPLAY\-BLK and CDROMREADALL}. 

\subsection{\cdrom\ capabilities}

Instead of just implementing some $ioctl$ calls, the interface in
\cdromc\ supplies the possibility to indicate the {\em capabilities\/}
of a \cdrom\ drive. This can be done by ORing any number of
capability-constants that are defined in \ucdrom\ at the registration
phase. Currently, the capabilities are any of:
$$
\halign{$#$\ \hfil&$/*$ \rm# $*/$\hfil\cr
CDC_CLOSE_TRAY& can close tray by software control\cr
CDC_OPEN_TRAY& can open tray\cr
CDC_LOCK& can lock and unlock the door\cr
CDC_SELECT_SPEED& can select speed, in units of $\sim$150\,kB/s\cr
CDC_SELECT_DISC& drive is juke-box\cr
CDC_MULTI_SESSION& can read sessions $>\rm1$\cr
CDC_MCN& can read Medium Catalog Number\cr
CDC_MEDIA_CHANGED& can report if disc has changed\cr
CDC_PLAY_AUDIO& can perform audio-functions (play, pause, etc)\cr
}
$$
The capability flag is declared $const$, to prevent drivers to
accidentally tamper with the contents. However, upon registration,
some (claimed) capability flags may be cleared if the supporting
function has not been implemented (see $register_cdrom()$ in
\cdromc). 

If you want to disable any of the capabilities, there is a special
flag register $<device>_dops.mask$ that may (temporarily) disable
certain capabilities. In the file \cdromc\ you will encounter many
constructions of the type 
$$\it
if\ (cdo\rightarrow capability \mathrel\& \mathord{\sim} cdo\rightarrow mask 
   \mathrel{\&} CDC_<capability>) \ldots
$$
The $mask$ could be set in the low-level driver code do disable
certain capabilities for special brands of the device that can't
perform the actions.  However, there is not (yet) and $ioctl$ to set
the mask\dots The reason is that I think it is better to control the
{\em behavior\/} rather than the {\em capabilities}.

\subsection{Options}

A final flag register controls the {\em behavior\/} of the \cdrom\
drives, in order to satisfy the different users's wishes, hopefully
independently of the ideas of the respectable author that happened to
have made the drive's support available to the \linux\ community. The
current behavior options are:
$$
\halign{$#$\ \hfil&$/*$ \rm# $*/$\hfil\cr
CDO_AUTO_CLOSE& try to close tray upon device $open()$\cr
CDO_AUTO_EJECT& try to open tray on last device $close()$\cr
CDO_USE_FFLAGS& use $file_pointer\rightarrow f_flags$ to indicate
 purpose for $open()$\cr
CDO_LOCK& try to lock door if device is opened\cr
CDO_CHECK_TYPE& ensure disc type is data if opened for data\cr
}
$$

The initial value of this register is $CDO_AUTO_CLOSE \mathrel|
CDO_USE_FFLAGS \mathrel| CDO_LOCK$, reflecting my own view on user
interface and software standards. Before you protest, there are two
new $ioctl$s implemented in \cdromc, that allow you to control the
behavior by software. These are:
$$
\halign{$#$\ \hfil&$/*$ \rm# $*/$\hfil\cr
CDROM_SET_OPTIONS& set options specified in $(int)\ arg$\cr
CDROM_CLEAR_OPTIONS& clear options specified in $(int)\ arg$\cr
}
$$
One option needs some more explanation: $CDO_USE_FFLAGS$. In the next
section we explain what the need for this option is.

\section{The need to know the purpose of opening}

Traditionally, Unix devices can be used in two different `modes',
either by reading/writing to the device file, or by issuing
controlling commands to the device, by the device's $ioctl()$
call. The problem with \cdrom\ drives, is that they can be used for
two entirely different purposes. One is to mount removable
file systems, \cdrom s, the other is to play audio CD's. Audio commands
are implemented entirely through $ioctl$s, presumably because the
first implementation (SUN?) has been such. In principle there is
nothing wrong with this, but a good control of the `CD player' demands
that the device can {\em always\/} be opened in order to give the
$ioctl$ commands, regardless of the state the drive is in. 

On the other hand, when used as a removable-media disc drive (what the
original purpose of \cdrom s is) we would like to make sure that the
disc drive is ready for operation upon opening the device. In the old
scheme, some \cdrom\ drivers don't do any integrity checking, resulting
in a number of i/o errors reported by the VFS to the kernel when an
attempt for mounting a \cdrom\ on an empty drive occurs. This is not a
particularly elegant way to find out that there is no \cdrom\ inserted;
it more-or-less looks like the old IBM-PC trying to read an empty floppy
drive for a couple of seconds, after which the system complains it
can't read from it. Nowadays we can {\em sense\/} the existence of a
removable medium in a drive, and we believe we should exploit that
fact. An integrity check on opening of the device, that verifies the
availability of a \cdrom\ and its correct type (data), would be
desirable.

These two ways of using a \cdrom\ drive, principally for data and
secondarily for playing audio discs, have different demands for the
behavior of the $open()$ call. Audio use simply wants to open the
device in order to get a file handle which is needed for issuing
$ioctl$ commands, while data use wants to open for correct and
reliable data transfer. The only way user programs can indicate what
their {\em purpose\/} of opening the device is, is trough the $flags$
parameter (see {\tt open(2)}). For \cdrom\ devices, these flags aren't
implemented (some drivers implement checking for write-related flags,
but this is not strictly necessary if the device file has correct
permission flags). Most option flags simply don't make sense to
\cdrom\ devices: $O_CREAT$, $O_NOCTTY$, $O_TRUNC$, $O_APPEND$, and
$O_SYNC$ have no meaning to a \cdrom. 

We therefore propose to use the flag $O_NONBLOCK$ as an indication
that the device is opened just for issuing $ioctl$
commands. Strictly, the meaning of $O_NONBLOCK$ is that opening and
subsequent calls to the device don't cause the calling process to
wait. We could interpret this as ``don't wait until someone has been
inserted some valid data-\cdrom.'' Thus, our proposal of the
implementation for the $open()$ call for \cdrom s is:
\begin{itemize}
\item If no other flags are set than $O_RDONLY$, the device is opened
for data transfer, and the return value will be 0 only upon successful
initialization of the transfer. The call may even induce some actions
on the \cdrom, such as closing the tray.  
\item If the option flag $O_NONBLOCK$ is set, opening will always be
successful, unless the whole device doesn't exist. The drive will take
no actions whatsoever. 
\end{itemize}

\subsection{And what about standards?}

You might hesitate to accept this proposal as is comes from the
\linux\ community, and not from some standardizing institute. What
about SUN, SGI, HP and all those other Unix and hardware vendors?
Well, these companies are in the lucky position that they generally
control both the hardware and software of their supported products,
and are large enough to set their own standard. They do not have to
deal with a dozen or more different, competing hardware
configurations.\footnote{Personally, I think that SUN's approach to
mounting \cdrom s is very good in origin: under Solaris a
volume-daemon automatically mounts a newly inserted \cdrom\ under {\tt
/cdrom/$<volume-name>$/}. In my opinion they should have pushed this
further and have {\em every\/} \cdrom\ on the local area network be
mounted at the similar location, i.e., no matter in which particular
machine you insert a \cdrom, it will always appear at the same
position in the directory tree, on every system. When I wanted to
implement such a user-program for \linux, I came across the
differences in behavior of the various drivers, and the need for an
$ioctl$ informing about media changes.}

We believe that using $O_NONBLOCK$ as indication for opening a device
for $ioctl$ commanding only, can be easily introduced in the \linux\
community. All the CD-player authors will have to be informed, we can
even send in our own patches to the programs. The use of $O_NONBLOCK$
has most likely no influence on the behavior of the CD-players on
other operating systems than \linux. Finally, a user can always revert
to old behavior by a call to $ioctl(file_descriptor, CDROM_CLEAR_OPTIONS,
CDO_USE_FFLAGS)$. 

\subsection{The preferred strategy of $open()$}

The routines in \cdromc\ are designed in such way, that a run-time
configuration of the behavior of \cdrom\ devices (of {\em any\/} type)
can be carried out, by the $CDROM_SET/CLEAR_OPTIONS$ $ioctls$. Thus, various
modes of operation can be set:
\begin{description}
\item[$CDO_AUTO_CLOSE \mathrel| CDO_USE_FFLAGS \mathrel| CDO_LOCK$]
This is the default setting. (With $CDO_CHECK_TYPE$ it will be better,
in the future.) If the device is not yet opened by any other process,
and it is opened for data ($O_NONBLOCK$ is not set) and the tray is
found open, an attempt to close the tray is made. Then, it is verified
that a disc is in the drive and, if $CDO_CHECK_TYPE$ is set, that its
type is `data mode 1.' Only if all tests are passed, the return value
is zero. The door is locked to prevent file system corruption. If
opened for audio ($O_NONBLOCK$ is set), no actions are taken and a
value of 0 will be returned.
\item[0] $Open()$ will always be successful, the option flags are
ignored. Neither actions are undertaken, nor any integrity checks are
made. 
\item[$CDO_AUTO_CLOSE \mathrel| CDO_AUTO_EJECT \mathrel| CDO_LOCK$]
This mimics the behavior of the current sbpcd-driver. The option flags
are ignored, the tray is closed on the first open, if
necessary. Similarly, the tray is opened on the last release, i.e., if
a \cdrom\ is unmounted, it is automatically ejected, such that the
user can replace it. 
\end{description}
We hope that these option can convince everybody (both driver
maintainers and user program developers) to adapt to the new cdrom
driver scheme and option flag interpretation. 

\section{Description of routines in \cdromc}

Only a few routines in \cdromc\ are exported to the drivers. In this
section we will treat these, as well as the functioning of the routines
that `take over' the interface to the kernel. The header file
belonging to \cdromc\ is called \ucdrom, but may be included in {\tt
cdrom.h} in the future.

\subsection{$struct\ file_operations\ cdrom_fops$}

The contents of this structure has been described in
section~\ref{cdrom.c}, and this structure should be used in
registering the block device to the kernel:
$$
register_blkdev(major, <name>, \&cdrom_fops);
$$

\subsection{$Int\ register_cdrom(int\ major, char * name, struct\
cdrom_device_ops\ * cdo)$}

Similar to registering $cdrom_fops$ to the kernel, the device
operations structure, as described in section~\ref{cdrom.c}, should be
registered to the general \cdrom\ interface:
$$
register_cdrom(major, <name>, \&<device>_dops);
$$
This function returns zero upon success, and non-zero upon failure. 

\subsection{$Int\ unregister_cdrom(int\ major, char * name)$}

Unregistering device $name$ with major number $major$ disconnects the
registered device-operation routines from the \cdrom\ interface.
This function returns zero upon success, and non-zero upon failure. 

\subsection{$Int\ cdrom_open(struct\ inode * ip, struct\ file * fp)$}

This function is not called directly by the low-level drivers, it is
listed in the standard $cdrom_fops$. If the VFS opens a file, this
function becomes active. A strategy logic is implemented in this
routine, taking care of all capabilities and options that are set in
the $cdrom_device_ops$ connected to the device. Then, the program flow is
transferred to the device_dependent $open()$ call. 

\subsection{$Void\ cdrom_release(struct\ inode *ip, struct\ file
*fp)$}

This function implements the reverse-logic of $cdrom_open()$, and then
calls the device-dependent $release()$ routine.  When the use-count
has reached 0, the allocated buffers in the are flushed by calls to
$sync_dev(dev)$ and $invalidate_buffers(dev)$.


\subsection{$Int\ cdrom_ioctl(struct\ inode *ip, struct\ file *fp,
                       unsigned\ int\ cmd, unsigned\ long\ arg)$}
\label{cdrom-ioctl}

This function handles all $ioctl$ requests for \cdrom\ devices in a
uniform way. The different calls fall into three categories: $ioctl$s
that can be directly implemented by device operations, ones that are
routed through the call $audio_ioctl()$, and the remaining ones, that
are presumable device-dependent. Generally, a negative return value
indicates an error. 

\subsubsection{Directly implemented $ioctl$s}
\label{ioctl-direct}

The following `old' \cdrom-$ioctl$s are implemented by directly
calling device-operations in $cdrom_device_ops$, if implemented and
not masked:
\begin{description}
\item[CDROMMULTISESSION] Requests the last session on a \cdrom.
\item[CDROMEJECT] Open tray. 
\item[CDROMCLOSETRAY] Close tray.
\item[CDROMEJECT_SW] If $arg\not=0$, set behavior to auto-close (close
tray on first open) and auto-eject (eject on last release), otherwise
set behavior to non-moving on $open()$ and $release()$ calls.
\item[CDROM_GET_MCN or CDROM_GET_UPC] Get the Medium Catalog Number from a CD.
\end{description}

\subsubsection{$Ioctl$s rooted through $audio_ioctl()$}
\label{ioctl-audio}

The following set of $ioctl$s are all implemented through a call to
the $cdrom_fops$ function $audio_ioctl()$. Memory checks and
allocation are performed in $cdrom_ioctl()$, and also sanitization of
address format ($CDROM_LBA$/$CDROM_MSF$) is done.
\begin{description}
\item[CDROMSUBCHNL] Get sub-channel data in argument $arg$ of type $struct\
cdrom_subchnl *{}$.
\item[CDROMREADTOCHDR] Read Table of Contents header, in $arg$ of type
$struct\ cdrom_tochdr *{}$. 
\item[CDROMREADTOCENTRY] Read a Table of Contents entry in $arg$ and
specified by $arg$ of type $struct\ cdrom_tocentry *{}$.
\item[CDROMPLAYMSF] Play audio fragment specified in Minute, Second,
Frame format, delimited by $arg$ of type $struct\ cdrom_msf *{}$.
\item[CDROMPLAYTRKIND] Play audio fragment in track-index format
delimited by $arg$ of type $struct cdrom_ti *{}$.
\item[CDROMVOLCTRL] Set volume specified by $arg$ of type $struct\
cdrom_volctrl *{}$.
\item[CDROMVOLREAD] Read volume into by $arg$ of type $struct\
cdrom_volctrl *{}$.
\item[CDROMSTART] Spin up disc.
\item[CDROMSTOP] Stop playback of audio fragment.
\item[CDROMPAUSE] Pause playback of audio fragment.
\item[CDROMRESUME] Resume playing.
\end{description}

\subsubsection{New $ioctl$s in \cdromc}

The following $ioctl$s have been introduced to allow user programs to
control the behavior of individual \cdrom\ devices. New $ioctl$
commands an be identified by their underscores in the name.
\begin{description}
\item[CDROM_SET_OPTIONS] Set options specified by $arg$. Returns the
option flag register after modification. Use  $arg = \rm0$ for reading
the current flags.
\item[CDROM_CLEAR_OPTIONS] Clear options specified by $arg$. Returns
  the option flag register after modification.
\item[CDROM_SELECT_SPEED] Select head-rate speed of disc specified as
  by $arg$. The value 0 means `auto-select', i.e., play audio discs at
  real time and data disc at maximum speed. The value $arg$ is
  checked against the maximum head rate of the drive found in
  the $cdrom_dops$.
\item[CDROM_SELECT_DISC] Select disc numbered $arg$ from a juke-box.
  First disc is numbered 0. The number $arg$ is checked against the
  maximum number of discs in the juke-box found in the $cdrom_dops$.
\item[CDROM_MEDIA_CHANGED] Returns 1 if a disc has been changed since
  the last call. Note that calls to cdrom_$media_changed$ by the VFS
  are treated by an independent queue, so both mechanisms will detect
  a media change once. Currently, \cdromc\ implements maximum 16 minors
  per major device.
\item[CDROM_DRIVE_STATUS] Returns the status of the drive by a call to
  $drive_status()$. Return values are as defined in section~\ref{drive
    status}. Note that this call doesn't return information on the
  current playing activity of the drive, this can be polled through an
  $ioctl$ call to $CDROMSUBCHNL$.
\item[CDROM_DISC_STATUS] Returns the type of the disc currently in the
  drive by a call to $disc_status()$. Return values are as defined in
  section~\ref{disc status}.
\end{description}

\subsubsection{Device dependent $ioct$s}

Finally, all other $ioctl$s are passed to the function $dev_ioctl()$,
if implemented. No memory allocation or verification is carried out. 

\subsection{How to update your driver}

We hope all \cdrom\ driver maintainers will understand the advantages
of re-routing the interface to the kernel though the new routines in
\cdromc. The following scheme should help you to update your
driver. It should not be too much work. We hope you want to take these
steps, in order to make the \linux\ \cdrom\ support more uniform and
more flexible.
\begin{enumerate}
\item Make a backup of your current driver. 
\item Get hold of the files \cdromc\ and \ucdrom, they should be in
the directory tree that came with this documentation. 
\item Include {\tt \char`\<linux/ucdrom.h>} just after {\tt cdrom.h}.
\item change the 3rd argument of $register_blkdev$ from
$\&<your-drive>_fops$ to $\&cdrom_fops$. 
\item Just after that line, add a line to register to the \cdrom\
routines: 
$$register_cdrom(major, <name>, <your-drive>_dops);$$
Similarly, add a call to $unregister_cdrom()$. 
\item Copy an example of the device-operations $struct$ to your source,
e.g., from {\tt cm206.c} $cm206_dops$, and change all entries to names
corresponding to your driver, or names you just happen to like. If
your driver doesn't support a certain function, make the entry
$NULL$. At the entry $capability$ you should list all capabilities
your drive could support, in principle. If your drive has a capability
that is not listed, please send me a message.
\item Implement all functions in your $<device>_dops$ structure,
according to prototypes listed in \ucdrom, and specifications given in
section~\ref{cdrom.c}. Most likely you have already implemented
the code in a large part, and you may just have to adapt the prototype
and return values. 
\item Rename your $<device>_ioctl()$ function to $audio_ioctl$ and
change the prototype a little. Remove entries listed in the first part
in section~\ref{cdrom-ioctl}, if your code was OK, this are just calls
to the routines you adapted in the previous step. 
\item You may remove all remaining memory checking code in the
$audio_ioctl()$ function that deal with audio commands (these are
listed in the second part of section~\ref{cdrom-ioctl}). There is no
need for memory allocation either, so most $case$s in the $switch$
statement look similar to:
$$
case\ CDROMREADTOCENTRY\colon
get_toc_entry\bigl((struct\ cdrom_tocentry *{})\ arg\bigr);
$$
\item All remaining $ioctl$ cases must be moved to a separate
function, $<device>_ioctl$, the device-dependent $ioctl$s. Note that
memory checking and allocation must be kept in this code!
\item Change the prototypes of $<device>_open()$ and
$<device>_release()$, and remove any strategic code (i.e., tray
movement, door locking, etc.). 
\item Try to recompile the drivers. We advice you to use modules, both
for {\tt cdrom.o} and your driver, as debugging is much easier this
way.
\end{enumerate} 

\section{Thanks}

Thanks to all the people involved. Thanks to Thomas Quinot, Jon Tombs,
Ken Pizzini, Eberhard M\"onkeberg and Andrew Kroll, the \linux\ 
\cdrom\ device driver developers that were kind enough to give
sugestions and criticisms during the writing. Finally of course, I
want to thank Linus Torvalds for making this possible in the first
place.

\end{document}

