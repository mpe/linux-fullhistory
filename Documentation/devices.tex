\documentstyle{article}
%
% Adopt somewhat reasonable margins, so it doesn't take a million
% pages to print... :-)  If you're actually putting this in print, you
% may wish to change these.
%
\oddsidemargin=0in
\textwidth=6.5in
\topmargin=0in
\headheight=0.5in
\headsep=0.25in
\textheight=7.5in
\footskip=0.75in
\footheight=0.5in
%
\begin{document}
\newcommand{\file}{\tt}			% Style to use for a filename
\newcommand{\hex}{\tt}			% Style to use for a hex number
\newcommand{\ud}{(Under development)}	% Abbreviation
\newcommand{\1}{\({}^1\)}
\newcommand{\2}{\({}^2\)}
\newcommand{\3}{\({}^3\)}
\newcommand{\4}{\({}^4\)}
\newlength{\dig}
\settowidth{\dig}{0}			% Get width of digits
\newcommand{\num}[2]{\makebox[#1\dig][r]{#2}}
\newcommand{\major}[4]{\num{3}{#1}#2 \> #3 \> #4 \\}
\newcommand{\minor}[3]{\> \> \num{3}{#1} \> {\file #2} \> #3 \\}
\newcommand{\minordots}{\> \> \> \dots \\}
\newenvironment{devicelist}%
 {\begin{tabbing}%
000--000 \= blockxxx \= 000 \= {\file /dev/crambamboli} \= foo \kill}%
 {\end{tabbing}}
\newcommand{\link}[4]{{\file #1} \> {\file #2} \> #3 \> #4 \\}
\newcommand{\vlink}[4]{{\file #1} \> {\em #2 \/} \> #3 \> #4 \\}
\newcommand{\node}[3]{{\file #1} \> #2 \> #3 \\}
\newenvironment{nodelist}%
 {\begin{tabbing}%
{\file /dev/crambamboli} \= {\file /proc/self/fd/99} \= symbolicxxx \=
foo \kill}%
 {\end{tabbing}}
%
\title{{\bf Linux Allocated Devices}}
\author{Maintained by H. Peter Anvin $<$hpa@zytor.com$>$}
\date{Last revised: May 12, 1996}
\maketitle
%
\noindent
This list is the successor to Rick Miller's Linux Device List, which
he stopped maintaining when he got busy with other things in 1993.  It
is a registry of allocated major device numbers, as well as the
recommended {\file /dev} directory nodes for these devices.

The latest version of this list is included with the Linux kernel
sources in \LaTeX\ and ASCII form.  In case of discrepancy, the
\LaTeX\ version is authoritative.

This document is included by reference into the Linux Filesystem
Standard (FSSTND).  The FSSTND is available via FTP from
tsx-11.mit.edu in the directory {\file
/pub/linux/docs/linux-standards/fsstnd}.

To have a major number allocated, or a minor number in situations
where that applies (e.g.\ busmice), please contact me with the
appropriate device information.  Also, if you have additional
information regarding any of the devices listed below, or if I have
made a mistake, I would greatly appreciate a note.

Allocations marked (68k/Amiga) apply to Linux/68k on the Amiga
platform only.  Allocations marked (68k/Atari) apply to Linux/68k on
the Atari platform only.

This document is in the public domain.  The author requests, however,
that semantically altered versions are not distributed without
permission of the author, assuming the author can be contacted without
an unreasonable effort.

In particular, please don't sent patches for this list to Linus, at
least not without contacting me first.

\section{Major numbers}

\begin{devicelist}
\major{ 0}{}{     }{Unnamed devices (e.g. non-device mounts)}
\major{ 1}{}{char }{Memory devices}
\major{  }{}{block}{RAM disk}
\major{ 2}{}{char }{Pseudo-TTY masters}
\major{  }{}{block}{Floppy disks}
\major{ 3}{}{char }{Pseudo-TTY slaves}
\major{  }{}{block}{First MFM, RLL and IDE hard disk/CD-ROM interface}
\major{ 4}{}{char }{TTY devices}
\major{ 5}{}{char }{Alternate TTY devices}
\major{ 6}{}{char }{Parallel printer devices}
\major{ 7}{}{char }{Virtual console access devices}
\major{  }{}{block}{Loopback devices}
\major{ 8}{}{block}{SCSI disk devices}
\major{ 9}{}{char }{SCSI tape devices}
\major{  }{}{block}{Metadisk (RAID) devices}
\major{10}{}{char }{Non-serial mice, misc features}
\major{11}{}{char }{Raw keyboard device}
\major{  }{}{block}{SCSI CD-ROM devices}
\major{12}{}{char }{QIC-02 tape}
\major{  }{}{block}{MSCDEX CD-ROM callback support}
\major{13}{}{char }{PC speaker}
\major{  }{}{block}{8-bit MFM/RLL/IDE controller}
\major{14}{}{char }{Sound card}
\major{  }{}{block}{BIOS harddrive callback support}
\major{15}{}{char }{Joystick}
\major{  }{}{block}{Sony CDU-31A/CDU-33A CD-ROM}
\major{16}{}{char }{Non-SCSI scanners}
\major{  }{}{block}{GoldStar CD-ROM}
\major{17}{}{char }{Chase serial card}
\major{  }{}{block}{Optics Storage CD-ROM}
\major{18}{}{char }{Chase serial card -- alternate devices}
\major{  }{}{block}{Sanyo CD-ROM}
\major{19}{}{char }{Cyclades serial card}
\major{  }{}{block}{Double compressed disk}
\major{20}{}{char }{Cyclades serial card -- alternate devices}
\major{  }{}{block}{Hitachi CD-ROM}
\major{21}{}{char }{Generic SCSI access}
\major{22}{}{char }{Digiboard serial card}
\major{  }{}{block}{Second IDE hard disk/CD-ROM interface}
\major{23}{}{char }{Digiboard serial card -- alternate devices}
\major{  }{}{block}{Mitsumi proprietary CD-ROM}
\major{24}{}{char }{Stallion serial card}
\major{  }{}{block}{Sony CDU-535 CD-ROM}
\major{25}{}{char }{Stallion serial card -- alternate devices}
\major{  }{}{block}{First Matsushita (Panasonic/SoundBlaster) CD-ROM}
\major{26}{}{char }{Quanta WinVision frame grabber}
\major{  }{}{block}{Second Matsushita (Panasonic/SoundBlaster) CD-ROM}
\major{27}{}{char }{QIC-117 tape}
\major{  }{}{block}{Third Matsushita (Panasonic/SoundBlaster) CD-ROM}
\major{28}{}{char }{Stallion serial card -- card programming}
\major{  }{}{char }{Atari SLM ACSI laser printer (68k/Atari)}
\major{  }{}{block}{Fourth Matsushita (Panasonic/SoundBlaster) CD-ROM}
\major{  }{}{block}{ACSI disk (68k/Atari)}
\major{29}{}{char }{Universal frame buffer}
\major{  }{}{block}{Aztech/Orchid/Okano/Wearnes CD-ROM}
\major{30}{}{char }{iBCS-2}
\major{  }{}{block}{Philips LMS-205 CD-ROM}
\major{31}{}{char }{MPU-401 MIDI}
\major{  }{}{block}{ROM/flash memory card}
\major{32}{}{char }{Specialix serial card}
\major{  }{}{block}{Philips LMS-206 CD-ROM}
\major{33}{}{char }{Specialix serial card -- alternate devices}
\major{  }{}{block}{Third IDE hard disk/CD-ROM interface}
\major{34}{}{char }{Z8530 HDLC driver}
\major{  }{}{block}{Fourth IDE hard disk/CD-ROM interface}
\major{35}{}{char }{tclmidi MIDI driver}
\major{  }{}{block}{Modular RAM disk}
\major{36}{}{char }{Netlink support}
\major{  }{}{block}{MCA ESDI hard disk}
\major{37}{}{char }{IDE tape}
\major{  }{}{block}{Zorro II ramdisk}
\major{38}{}{char }{Myricom PCI Myrinet board}
\major{  }{}{block}{Reserved for Linux/AP+}
\major{39}{}{char }{ML-16P experimental I/O board}
\major{  }{}{block}{Reserved for Linux/AP+}
\major{40}{}{char }{Matrox Meteor frame grabber}
\major{  }{}{block}{Syquest EZ135 removable drive}
\major{41}{}{char }{Yet Another Micro Monitor}
\major{42}{}{}{Demo/sample use}
\major{43}{}{char }{isdn4linux virtual modem}
\major{44}{}{char }{isdn4linux virtual modem -- alternate devices}
\major{45}{}{char }{isdn4linux ISDN BRI driver}
\major{46}{}{char }{Comtrol Rocketport serial card}
\major{47}{}{char }{Comtrol Rocketport serial card -- alternate devices}
\major{48}{}{char }{SDL RISCom serial card}
\major{49}{}{char }{SDL RISCom serial card -- alternate devices}
\major{50}{}{char }{Reserved for GLINT}
\major{51}{}{char }{Baycom radio modem}
\major{52}{--59}{}{Unallocated}
\major{60}{--63}{}{Local/experimental use}
\major{64}{--119}{}{Unallocated}
\major{120}{--127}{}{Local/experimental use}
\major{128}{--239}{}{Unallocated}
\major{240}{--254}{}{Local/experimental use}
\major{255}{}{}{Reserved}
\end{devicelist}

\section{Minor numbers}


\begin{devicelist}
\major{0}{}{}{Unnamed devices (e.g. non-device mounts)}
	\minor{0}{}{reserved as null device number}
\end{devicelist}

\begin{devicelist}
\major{1}{}{char}{Memory devices}
	\minor{1}{/dev/mem}{Physical memory access}
	\minor{2}{/dev/kmem}{Kernel virtual memory access}
	\minor{3}{/dev/null}{Null device}
	\minor{4}{/dev/port}{I/O port access}
	\minor{5}{/dev/zero}{Null byte source}
	\minor{6}{/dev/core}{OBSOLETE -- should be a link to {\file /proc/kcore}}
	\minor{7}{/dev/full}{Returns ENOSPC on write}
	\minor{8}{/dev/random}{Nondeterministic random number generator}
	\minor{9}{/dev/urandom}{Less secure, but faster random number generator}
\\
\major{}{}{block}{RAM disk}
	\minor{0}{/dev/ram0}{First RAM disk}
	\minordots
	\minor{7}{/dev/ram7}{Eighth RAM disk}
	\minor{250}{/dev/initrd}{Initial RAM disk}
\end{devicelist}

\noindent
Earlier kernels had {\file /dev/ramdisk} (1, 1) here.  {\file /dev/initrd}
refers to a RAM disk which was preloaded by the boot loader.

\begin{devicelist}
\major{2}{}{char}{Pseudo-TTY masters}
	\minor{0}{/dev/ptyp0}{First PTY master}
	\minor{1}{/dev/ptyp1}{Second PTY master}
	\minordots
	\minor{255}{/dev/ptyef}{256th PTY master}
\end{devicelist}

\noindent
Pseudo-TTY's are named as follows:
\begin{itemize}
\item Masters are {\file pty}, slaves are {\file tty};
\item the fourth letter is one of {\file pqrstuvwxyzabcde} indicating
the 1st through 16th series of 16 pseudo-ttys each, and
\item the fifth letter is one of {\file 0123456789abcdef} indicating
the position within the series.
\end{itemize}

\begin{devicelist}
\major{}{}{block}{Floppy disks}
	\minor{0}{/dev/fd0}{Controller 1, drive 1 autodetect}
	\minor{1}{/dev/fd1}{Controller 1, drive 2 autodetect}
	\minor{2}{/dev/fd2}{Controller 1, drive 3 autodetect}
	\minor{3}{/dev/fd3}{Controller 1, drive 4 autodetect}
	\minor{128}{/dev/fd4}{Controller 2, drive 1 autodetect}
	\minor{129}{/dev/fd5}{Controller 2, drive 2 autodetect}
	\minor{130}{/dev/fd6}{Controller 2, drive 3 autodetect}
	\minor{131}{/dev/fd7}{Controller 2, drive 4 autodetect}
\\
\major{}{}{}{To specify format, add to the autodetect device number}
	\minor{  0}{/dev/fd?}{Autodetect format}
	\minor{  4}{/dev/fd?d360}{5.25" \num{4}{360}K in a \num{4}{360}K drive\1}
	\minor{ 20}{/dev/fd?h360}{5.25" \num{4}{360}K in a 1200K drive\1}
	\minor{ 48}{/dev/fd?h410}{5.25" \num{4}{410}K in a 1200K drive}
	\minor{ 64}{/dev/fd?h420}{5.25" \num{4}{420}K in a 1200K drive}
	\minor{ 24}{/dev/fd?h720}{5.25" \num{4}{720}K in a 1200K drive}
	\minor{ 80}{/dev/fd?h880}{5.25" \num{4}{880}K in a 1200K drive\1}
	\minor{  8}{/dev/fd?h1200}{5.25" 1200K in a 1200K drive\1}
	\minor{ 40}{/dev/fd?h1440}{5.25" 1440K in a 1200K drive\1}
	\minor{ 56}{/dev/fd?h1476}{5.25" 1476K in a 1200K drive}
	\minor{ 72}{/dev/fd?h1494}{5.25" 1494K in a 1200K drive}
	\minor{ 92}{/dev/fd?h1600}{5.25" 1600K in a 1200K drive\1}
	\minor{}{}{}
	\minor{ 12}{/dev/fd?u360}{3.5" \num{4}{360}K Double Density}
	\minor{ 16}{/dev/fd?u720}{3.5" \num{4}{720}K Double Density\1}
	\minor{120}{/dev/fd?u800}{3.5" \num{4}{800}K Double Density\2}
	\minor{ 52}{/dev/fd?u820}{3.5" \num{4}{820}K Double Density}
	\minor{ 68}{/dev/fd?u830}{3.5" \num{4}{830}K Double Density}
	\minor{ 84}{/dev/fd?u1040}{3.5" 1040K Double Density\1}
	\minor{ 88}{/dev/fd?u1120}{3.5" 1120K Double Density\1}
	\minor{ 28}{/dev/fd?u1440}{3.5" 1440K High Density\1}
	\minor{124}{/dev/fd?u1600}{3.5" 1600K High Density\1}
	\minor{ 44}{/dev/fd?u1680}{3.5" 1680K High Density\3}
	\minor{ 60}{/dev/fd?u1722}{3.5" 1722K High Density}
	\minor{ 76}{/dev/fd?u1743}{3.5" 1743K High Density}
	\minor{ 96}{/dev/fd?u1760}{3.5" 1760K High Density}
	\minor{116}{/dev/fd?u1840}{3.5" 1840K High Density\3}
	\minor{100}{/dev/fd?u1920}{3.5" 1920K High Density\1}
	\minor{ 32}{/dev/fd?u2880}{3.5" 2880K Extra Density\1}
	\minor{104}{/dev/fd?u3200}{3.5" 3200K Extra Density}
	\minor{108}{/dev/fd?u3520}{3.5" 3520K Extra Density}
	\minor{112}{/dev/fd?u3840}{3.5" 3840K Extra Density\1}
	\minor{}{}{}
	\minor{36}{/dev/fd?CompaQ}{Compaq 2880K drive; probably obsolete}
\\
\major{}{}{}{\1 Autodetectable format}
\major{}{}{}{\2 Autodetectable format in a Double Density (720K) drive only}
\major{}{}{}{\3 Autodetectable format in a High Density (1440K) drive only}
\end{devicelist}

NOTE: The letter in the device name ({\file d}, {\file q}, {\file h}
or {\file u}) signifies the type of drive supported: 5.25" Double
Density ({\file d}), 5.25" Quad Density ({\file q}), 5.25" High
Density ({\file h}) or 3.5" (any type, {\file u}).  The capital
letters {\file D}, {\file H}, or {\file E} for the 3.5" models have
been deprecated, since the drive type is insignificant for these devices.

\begin{devicelist}
\major{3}{}{char}{Pseudo-TTY slaves}
	\minor{0}{/dev/ttyp0}{First PTY slave}
	\minor{1}{/dev/ttyp1}{Second PTY slave}
	\minordots
	\minor{255}{/dev/ttyef}{256th PTY slave}
\\
\major{}{}{block}{First MFM, RLL and IDE hard disk/CD-ROM interface}
	\minor{0}{/dev/hda}{Master: whole disk (or CD-ROM)}
	\minor{64}{/dev/hdb}{Slave: whole disk (or CD-ROM)}
\\
\major{}{}{}{For partitions, add to the whole disk device number}
	\minor{0}{/dev/hd?}{Whole disk}
	\minor{1}{/dev/hd?1}{First partition}
	\minor{2}{/dev/hd?2}{Second partition}
	\minordots
	\minor{63}{/dev/hd?63}{63rd partition}
\end{devicelist}

\noindent
For Linux/i386, partitions 1-4 are the primary partitions, partitions
5 and up are logical partitions.  Other versions of Linux use
partitioning schemes appropriate to their respective architectures.

\begin{devicelist}
\major{ 4}{}{char }{TTY devices}
	\minor{0}{/dev/console}{Console device}
	\minor{1}{/dev/tty1}{First virtual console}
	\minordots
	\minor{63}{/dev/tty63}{63rd virtual console}
	\minor{64}{/dev/ttyS0}{First serial port}
	\minordots
	\minor{127}{/dev/ttyS63}{64th serial port}
	\minor{128}{/dev/ptyp0}{First pseudo-tty master}
	\minordots
	\minor{191}{/dev/ptysf}{64th pseudo-tty master}
	\minor{192}{/dev/ttyp0}{First pseudo-tty slave}
	\minordots
	\minor{255}{/dev/ttysf}{64th pseudo-tty slave}
\end{devicelist}

\noindent
For compatibility with previous versions of Linux, the first 64 PTYs
are replicated under this device number.  This use will be obsolescent
with the release of Linux 2.0 and may be removed in a future version
of Linux.

\begin{devicelist}
\major{ 5}{}{char }{Alternate TTY devices}
	\minor{0}{/dev/tty}{Current TTY device}
	\minor{64}{/dev/cua0}{Callout device corresponding to {\file ttyS0}}
	\minordots
	\minor{127}{/dev/cua63}{Callout device corresponding to {\file ttyS63}}
\end{devicelist}

\begin{devicelist}
\major{ 6}{}{char }{Parallel printer devices}
	\minor{0}{/dev/lp0}{First parallel printer ({\hex 0x3bc})}
	\minor{1}{/dev/lp1}{Second parallel printer ({\hex 0x378})}
	\minor{2}{/dev/lp2}{Third parallel printer ({\hex 0x278})}
\end{devicelist}

\noindent
Not all computers have the {\hex 0x3bc} parallel port, hence the
"first" printer may be either {\file /dev/lp0} or {\file /dev/lp1}.

\begin{devicelist}
\major{ 7}{}{char }{Virtual console access devices}
	\minor{0}{/dev/vcs}{Current vc text access}
	\minor{1}{/dev/vcs1}{tty1 text access}
	\minordots
	\minor{63}{/dev/vcs63}{tty63 text access}
	\minor{128}{/dev/vcsa}{Current vc text/attribute access}
	\minor{129}{/dev/vcsa1}{tty1 text/attribute access}
	\minordots
	\minor{191}{/dev/vcsa63}{tty63 text/attribute access}
\end{devicelist}

\noindent
NOTE: These devices permit both read and write access.

\begin{devicelist}
\major{  }{}{block}{Loopback devices}
	\minor{0}{/dev/loop0}{First loopback device}
	\minor{1}{/dev/loop1}{Second loopback device}
	\minordots
\end{devicelist}

\begin{devicelist}
\major{ 8}{}{block}{SCSI disk devices}
	\minor{0}{/dev/sda}{First SCSI disk whole disk}
	\minor{16}{/dev/sdb}{Second SCSI disk whole disk}
	\minor{32}{/dev/sdc}{Third SCSI disk whole disk}
	\minordots
	\minor{240}{/dev/sdp}{Sixteenth SCSI disk whole disk}
\end{devicelist}

\noindent
Partitions are handled in the same way as for IDE disks (see major
number 3) except that the partition limit is 15 rather than 63 per
disk.

\begin{devicelist}
\major{ 9}{}{char }{SCSI tape devices}
	\minor{0}{/dev/st0}{First SCSI tape, mode 0}
	\minor{1}{/dev/st1}{Second SCSI tape, mode 0}
	\minordots
	\minor{32}{/dev/st0l}{First SCSI tape, mode 1}
	\minor{33}{/dev/st1l}{Second SCSI tape, mode 1}
	\minordots
	\minor{64}{/dev/st0m}{First SCSI tape, mode 2}
	\minor{65}{/dev/st1m}{Second SCSI tape, mode 2}
	\minordots
	\minor{96}{/dev/st0a}{First SCSI tape, mode 3}
	\minor{97}{/dev/st1a}{Second SCSI tape, mode 4}
	\minordots
	\minor{128}{/dev/nst0}{First SCSI tape, mode 0, no rewind}
	\minor{129}{/dev/nst1}{Second SCSI tape, mode 0, no rewind}
	\minordots
	\minor{160}{/dev/nst0l}{First SCSI tape, mode 1, no rewind}
	\minor{161}{/dev/nst1l}{Second SCSI tape, mode 1, no rewind}
	\minordots
	\minor{192}{/dev/nst0m}{First SCSI tape, mode 2, no rewind}
	\minor{193}{/dev/nst1m}{Second SCSI tape, mode 2, no rewind}
	\minordots
	\minor{224}{/dev/nst0a}{First SCSI tape, mode 3, no rewind}
	\minor{225}{/dev/nst1a}{Second SCSI tape, mode 3, no rewind}
	\minordots
\end{devicelist}

\noindent
``No rewind'' refers to the omission of the default automatic rewind
on device close.  The {\file MTREW} or {\file MTOFFL} ioctl()s can be
used to rewind the tape regardless of the device used to access it.

\begin{devicelist}
\major{  }{}{block}{Metadisk (RAID) devices}
	\minor{0}{/dev/md0}{First metadisk group}
	\minor{1}{/dev/md1}{Second metadisk group}
	\minordots
\end{devicelist}

\noindent
The metadisk driver is used to span a filesystem across multiple
physical disks.

\begin{devicelist}
\major{10}{}{char }{Non-serial mice, misc features}
	\minor{0}{/dev/logibm}{Logitech bus mouse}
	\minor{1}{/dev/psaux}{PS/2-style mouse port}
	\minor{2}{/dev/inportbm}{Microsoft Inport bus mouse}
	\minor{3}{/dev/atibm}{ATI XL bus mouse}
	\minor{4}{/dev/jbm}{J-mouse}
	\minor{4}{/dev/amigamouse}{Amiga mouse (68k/Amiga)}
	\minor{5}{/dev/atarimouse}{Atari mouse}
	\minor{6}{/dev/sunmouse}{Sun mouse}
	\minor{7}{/dev/amigamouse1}{Second Amiga mouse}
	\minor{128}{/dev/beep}{Fancy beep device}
	\minor{129}{/dev/modreq}{Kernel module load request}
	\minor{130}{/dev/watchdog}{Watchdog timer port}
	\minor{131}{/dev/temperature}{Machine internal temperature}
	\minor{132}{/dev/hwtrap}{Hardware fault trap}
	\minor{133}{/dev/exttrp}{External device trap}
	\minor{134}{/dev/apm\_bios}{Advanced Power Management BIOS}
	\minor{135}{/dev/rtc}{Real Time Clock}
	\minor{136}{/dev/qcam0}{QuickCam on {\file lp0}}
	\minor{137}{/dev/qcam1}{QuickCam on {\file lp1}}
	\minor{138}{/dev/qcam2}{QuickCam on {\file lp2}}
\end{devicelist}

\noindent
The loopback devices are used to mount filesystems not associated with
block devices.  The binding to the loopback devices is usually handled
by {\bf mount}(8).

\begin{devicelist}
\major{11}{}{char }{Raw keyboard device}
	\minor{0}{/dev/kbd}{Raw keyboard device}
\end{devicelist}

\noindent
The raw keyboard device is used on Linux/SPARC only.

\begin{devicelist}
\major{  }{}{block}{SCSI CD-ROM devices}
	\minor{0}{/dev/sr0}{First SCSI CD-ROM}
	\minor{1}{/dev/sr1}{Second SCSI CD-ROM}
	\minordots
\end{devicelist}

\noindent
The prefix {\file /dev/scd} instead of {\file /dev/sr} has been used
as well, and might make more sense.

\begin{devicelist}
\major{12}{}{char }{QIC-02 tape}
	\minor{2}{/dev/ntpqic11}{QIC-11, no rewind-on-close}
	\minor{3}{/dev/tpqic11}{QIC-11, rewind-on-close}
	\minor{4}{/dev/ntpqic24}{QIC-24, no rewind-on-close}
	\minor{5}{/dev/tpqic24}{QIC-24, rewind-on-close}
	\minor{6}{/dev/ntpqic120}{QIC-120, no rewind-on-close}
	\minor{7}{/dev/tpqic120}{QIC-120, rewind-on-close}
	\minor{8}{/dev/ntpqic150}{QIC-150, no rewind-on-close}
	\minor{9}{/dev/tpqic150}{QIC-150, rewind-on-close}
\end{devicelist}

\noindent
The device names specified are proposed -- if there are ``standard''
names for these devices, please let me know.

\begin{devicelist}
\major{  }{}{block}{MSCDEX CD-ROM callback support}
	\minor{0}{/dev/dos\_cd0}{First MSCDEX CD-ROM}
	\minor{1}{/dev/dos\_cd1}{Second MSCDEX CD-ROM}
	\minordots
\end{devicelist}

\begin{devicelist}
\major{13}{}{char }{PC speaker}
	\minor{0}{/dev/pcmixer}{Emulates {\file /dev/mixer}}
	\minor{3}{/dev/pcsp}{Emulates {\file /dev/dsp} (8-bit)}
	\minor{4}{/dev/pcaudio}{Emulates {\file /dev/audio}}
	\minor{5}{/dev/pcsp16}{Emulates {\file /dev/dsp} (16-bit)}
\\
\major{  }{}{block}{8-bit MFM/RLL/IDE controller}
	\minor{0}{/dev/xda}{First XT disk whole disk}
	\minor{64}{/dev/xdb}{Second XT disk whole disk}
\end{devicelist}

\noindent
Partitions are handled in the same way as for IDE disks (see major
number 3).

\begin{devicelist}
\major{14}{}{char }{Sound card}
	\minor{0}{/dev/mixer}{Mixer control}
	\minor{1}{/dev/sequencer}{Audio sequencer}
	\minor{2}{/dev/midi00}{First MIDI port}
	\minor{3}{/dev/dsp}{Digital audio}
	\minor{4}{/dev/audio}{Sun-compatible digital audio}
	\minor{6}{/dev/sndstat}{Sound card status information}
	\minor{8}{/dev/sequencer2}{Sequencer -- alternate device}
	\minor{16}{/dev/mixer1}{Second soundcard mixer control}
	\minor{17}{/dev/patmgr0}{Sequencer patch manager}
	\minor{18}{/dev/midi01}{Second MIDI port}
	\minor{19}{/dev/dsp1}{Second soundcard digital audio}
	\minor{20}{/dev/audio1}{Second soundcard Sun digital audio}
	\minor{33}{/dev/patmgr1}{Sequencer patch manager}
	\minor{34}{/dev/midi02}{Third MIDI port}
	\minor{50}{/dev/midi03}{Fourth MIDI port}
\\
\major{  }{}{block}{BIOS harddrive callback support}
	\minor{0}{/dev/dos\_hda}{First BIOS harddrive whole disk}
	\minor{64}{/dev/dos\_hdb}{Second BIOS harddrive whole disk}
	\minor{128}{/dev/dos\_hdc}{Third BIOS harddrive whole disk}
	\minor{192}{/dev/dos\_hdd}{Fourth BIOS harddrive whole disk}
\end{devicelist}

\noindent
Partitions are handled in the same way as for IDE disks (see major
number 3).

\begin{devicelist}
\major{15}{}{char }{Joystick}
	\minor{0}{/dev/js0}{First analog joystick}
	\minor{1}{/dev/js1}{Second analog joystick}
	\minordots
	\minor{128}{/dev/djs0}{First digital joystick}
	\minor{129}{/dev/djs1}{Second digital joystick}
	\minordots
\\
\major{  }{}{block}{Sony CDU-31A/CDU-33A CD-ROM}
	\minor{0}{/dev/sonycd}{Sony CDU-31A CD-ROM}
\end{devicelist}

\begin{devicelist}
\major{16}{}{char }{Non-SCSI scanners}
	\minor{0}{/dev/gs4500}{Genius 4500 handheld scanner}
\\
\major{  }{}{block}{GoldStar CD-ROM}
	\minor{0}{/dev/gscd}{GoldStar CD-ROM}
\end{devicelist}

\begin{devicelist}
\major{17}{}{char }{Chase serial card}
	\minor{0}{/dev/ttyH0}{First Chase port}
	\minor{1}{/dev/ttyH1}{Second Chase port}
	\minordots
\\
\major{  }{}{block}{Optics Storage CD-ROM}
	\minor{0}{/dev/optcd}{Optics Storage CD-ROM}
\end{devicelist}

\begin{devicelist}
\major{18}{}{char }{Chase serial card -- alternate devices}
	\minor{0}{/dev/cuh0}{Callout device corresponding to {\file ttyH0}}
	\minor{1}{/dev/cuh1}{Callout device corresponding to {\file ttyH1}}
	\minordots
\\
\major{  }{}{block}{Sanyo CD-ROM}
	\minor{0}{/dev/sjcd}{Sanyo CD-ROM}
\end{devicelist}

\begin{devicelist}
\major{19}{}{char }{Cyclades serial card}
	\minor{0}{/dev/ttyC0}{First Cyclades port}
	\minordots
	\minor{31}{/dev/ttyC31}{32nd Cyclades port}
\\
\major{  }{}{block}{``Double'' compressed disk}
	\minor{0}{/dev/double0}{First compressed disk}
	\minordots
	\minor{7}{/dev/double7}{Eighth compressed disk}
	\minor{128}{/dev/cdouble0}{Mirror of first compressed disk}
	\minordots
	\minor{135}{/dev/cdouble7}{Mirror of eighth compressed disk}
\end{devicelist}

\noindent
See the Double documentation for an explanation of the ``mirror'' devices.

\begin{devicelist}
\major{20}{}{char }{Cyclades serial card -- alternate devices}
	\minor{0}{/dev/cub0}{Callout device corresponding to {\file ttyC0}}
	\minordots
	\minor{31}{/dev/cub31}{Callout device corresponding to {\file ttyC31}}
\\
\major{  }{}{block}{Hitachi CD-ROM}
	\minor{0}{/dev/hitcd}{Hitachi CD-ROM}
\end{devicelist}

\begin{devicelist}
\major{21}{}{char }{Generic SCSI access}
	\minor{0}{/dev/sg0}{First generic SCSI device}
	\minor{1}{/dev/sg1}{Second generic SCSI device}
	\minordots
\end{devicelist}

\begin{devicelist}
\major{22}{}{char }{Digiboard serial card}
	\minor{0}{/dev/ttyD0}{First Digiboard port}
	\minor{1}{/dev/ttyD1}{Second Digiboard port}
	\minordots
\major{  }{}{block}{Second IDE hard disk/CD-ROM interface}
	\minor{0}{/dev/hdc}{Master: whole disk (or CD-ROM)}
	\minor{64}{/dev/hdd}{Slave: whole disk (or CD-ROM)}
\end{devicelist}

\noindent
Partitions are handled the same way as for the first interface (see
major number 3).

\begin{devicelist}
\major{23}{}{char }{Digiboard serial card -- alternate devices}
	\minor{0}{/dev/cud0}{Callout device corresponding to {\file ttyD0}}
	\minor{1}{/dev/cud1}{Callout device corresponding to {\file ttyD1}}
	\minordots
\major{  }{}{block}{Mitsumi proprietary CD-ROM}
	\minor{0}{/dev/mcd}{Mitsumi CD-ROM}
\end{devicelist}

\begin{devicelist}\
\major{24}{}{char }{Stallion serial card}
	\minor{0}{/dev/ttyE0}{Stallion port 0 board 0}
	\minor{1}{/dev/ttyE1}{Stallion port 1 board 0}
	\minordots
	\minor{64}{/dev/ttyE64}{Stallion port 0 board 1}
	\minor{65}{/dev/ttyE65}{Stallion port 1 board 1}
	\minordots
	\minor{128}{/dev/ttyE128}{Stallion port 0 board 2}
	\minor{129}{/dev/ttyE129}{Stallion port 1 board 2}
	\minordots
	\minor{192}{/dev/ttyE192}{Stallion port 0 board 3}
	\minor{193}{/dev/ttyE193}{Stallion port 1 board 3}
	\minordots
\\
\major{  }{}{block}{Sony CDU-535 CD-ROM}
	\minor{0}{/dev/cdu535}{Sony CDU-535 CD-ROM}
\end{devicelist}

\begin{devicelist}
\major{25}{}{char }{Stallion serial card -- alternate devices}
	\minor{0}{/dev/cue0}{Callout device corresponding to {\file ttyE0}}
	\minor{1}{/dev/cue1}{Callout device corresponding to {\file ttyE1}}
	\minordots
	\minor{64}{/dev/cue64}{Callout device corresponding to {\file ttyE64}}
	\minor{65}{/dev/cue65}{Callout device corresponding to {\file ttyE65}}
	\minordots
	\minor{128}{/dev/cue128}{Callout device corresponding to {\file ttyE128}}
	\minor{129}{/dev/cue129}{Callout device corresponding to {\file ttyE129}}
	\minordots
	\minor{192}{/dev/cue192}{Callout device corresponding to {\file ttyE192}}
	\minor{193}{/dev/cue193}{Callout device corresponding to {\file ttyE193}}
	\minordots
\\
\major{  }{}{block}{First Matsushita (Panasonic/SoundBlaster) CD-ROM}
	\minor{0}{/dev/sbpcd0}{Panasonic CD-ROM controller 0 unit 0}
	\minor{1}{/dev/sbpcd1}{Panasonic CD-ROM controller 0 unit 1}
	\minor{2}{/dev/sbpcd2}{Panasonic CD-ROM controller 0 unit 2}
	\minor{3}{/dev/sbpcd3}{Panasonic CD-ROM controller 0 unit 3}
\end{devicelist}

\begin{devicelist}
\major{26}{}{char }{Quanta WinVision frame grabber}
	\minor{0}{/dev/wvisfgrab}{Quanta WinVision frame grabber}
\\
\major{  }{}{block}{Second Matsushita (Panasonic/SoundBlaster) CD-ROM}
	\minor{0}{/dev/sbpcd4}{Panasonic CD-ROM controller 1 unit 0}
	\minor{1}{/dev/sbpcd5}{Panasonic CD-ROM controller 1 unit 1}
	\minor{2}{/dev/sbpcd6}{Panasonic CD-ROM controller 1 unit 2}
	\minor{3}{/dev/sbpcd7}{Panasonic CD-ROM controller 1 unit 3}
\end{devicelist}

\begin{devicelist}
\major{27}{}{char }{QIC-117 tape}
	\minor{0}{/dev/rft0}{Unit 0, rewind-on-close}
	\minor{1}{/dev/rft1}{Unit 1, rewind-on-close}
	\minor{2}{/dev/rft2}{Unit 2, rewind-on-close}
	\minor{3}{/dev/rft3}{Unit 3, rewind-on-close}
	\minor{4}{/dev/nrft0}{Unit 0, no rewind-on-close}
	\minor{5}{/dev/nrft1}{Unit 1, no rewind-on-close}
	\minor{6}{/dev/nrft2}{Unit 2, no rewind-on-close}
	\minor{7}{/dev/nrft3}{Unit 3, no rewind-on-close}
\\
\major{  }{}{block}{Third Matsushita (Panasonic/SoundBlaster) CD-ROM}
	\minor{0}{/dev/sbpcd8}{Panasonic CD-ROM controller 2 unit 0}
	\minor{1}{/dev/sbpcd9}{Panasonic CD-ROM controller 2 unit 1}
	\minor{2}{/dev/sbpcd10}{Panasonic CD-ROM controller 2 unit 2}
	\minor{3}{/dev/sbpcd11}{Panasonic CD-ROM controller 2 unit 3}
\end{devicelist}

\begin{devicelist}
\major{28}{}{char }{Stallion serial card -- card programming}
	\minor{0}{/dev/staliomem0}{First Stallion I/O card memory}
	\minor{1}{/dev/staliomem1}{Second Stallion I/O card memory}
	\minor{2}{/dev/staliomem2}{Third Stallion I/O card memory}
	\minor{3}{/dev/staliomem3}{Fourth Stallion I/O card memory}
\\
\major{  }{}{char }{Atari SLM ACSI laser printer (68k/Atari)}
	\minor{0}{/dev/slm0}{First SLM laser printer}
	\minor{1}{/dev/slm1}{Second SLM laser printer}
	\minordots
\\
\major{  }{}{block}{Fourth Matsushita (Panasonic/SoundBlaster) CD-ROM}
	\minor{0}{/dev/sbpcd12}{Panasonic CD-ROM controller 3 unit 0}
	\minor{1}{/dev/sbpcd13}{Panasonic CD-ROM controller 3 unit 1}
	\minor{2}{/dev/sbpcd14}{Panasonic CD-ROM controller 3 unit 2}
	\minor{3}{/dev/sbpcd15}{Panasonic CD-ROM controller 3 unit 3}
\\
\major{  }{}{block}{ACSI disk/CD-ROM (68k/Atari)}
	\minor{0}{/dev/ada}{First ACSI disk whole disk}
	\minor{16}{/dev/adb}{Second ACSI disk whole disk}
	\minor{32}{/dev/adc}{Third ACSI disk whole disk}
	\minordots
	\minor{240}{/dev/adp}{Sixteenth ACSI disk whole disk}
\end{devicelist}

\noindent
Partitions are handled in the same way as for IDE disks (see major
number 3) except that the partition limit is 15 rather than 63 per
disk (same as SCSI.)

\begin{devicelist}
\major{29}{}{char }{Universal frame buffer}
	\minor{0}{/dev/fb0}{First frame buffer}
	\minor{1}{/dev/fb0autodetect}{}
	\minor{24}{/dev/fb0user0}{}
	\minordots
	\minor{31}{/dev/fb0user7}{}
	\minor{32}{/dev/fb1}{Second frame buffer}
	\minor{33}{/dev/fb1autodetect}{}
	\minor{56}{/dev/fb1user0}{}
	\minordots
	\minor{63}{/dev/fb1user7}{}
	\minordots
\end{devicelist}

\noindent
The universal frame buffer device is currently supported only on
Linux/68k and Linux/SPARC.  The plain device accesses the frame
buffer at current resolution (Linux/68k calls this file {\file
current}, e.g. {\file /dev/fb0current}); the {\file autodetect} one at
bootup (default) resolution.  Minor numbers 2--23 within each frame
buffer assignment are used for specific device-dependent resolutions.
There appears to be no standard naming for these devices.  Finally,
24--31 within each device are reserved for user-selected modes,
usually entered at boot time.  Currently only Linux/68k uses the
mode-specific devices.

\begin{devicelist}
\major{  }{}{block}{Aztech/Orchid/Okano/Wearnes CD-ROM}
	\minor{0}{/dev/aztcd}{Aztech CD-ROM}
\end{devicelist}

\begin{devicelist}
\major{30}{}{char }{iBCS-2 compatibility devices}
	\minor{0}{/dev/socksys}{Socket access}
	\minor{1}{/dev/spx}{SVR3 local X interface}
	\minor{2}{/dev/inet/arp}{Network access}
	\minor{2}{/dev/inet/icmp}{Network access}
	\minor{2}{/dev/inet/ip}{Network access}
	\minor{2}{/dev/inet/udp}{Network access}
	\minor{2}{/dev/inet/tcp}{Network access}
\end{devicelist}

\noindent
iBCS-2 requires {\file /dev/nfsd} to be a link to {\file /dev/socksys}
and {\file /dev/X0R} to be a link to {\file /dev/null}.

\begin{devicelist}
\major{  }{}{block}{Philips LMS CM-205 CD-ROM}
	\minor{0}{/dev/cm205cd}{Philips LMS CM-205 CD-ROM}
\end{devicelist}

\noindent
{\file /dev/lmscd} is an older name for this drive.  This driver does
not work with the CM-205MS CD-ROM.

\begin{devicelist}
\major{31}{}{char }{MPU-401 MIDI}
	\minor{0}{/dev/mpu401data}{MPU-401 data port}
	\minor{1}{/dev/mpu401stat}{MPU-401 status port}
\\
\major{  }{}{block}{ROM/flash memory card}
	\minor{0}{/dev/rom0}{First ROM card (rw)}
	\minordots
	\minor{7}{/dev/rom7}{Eighth ROM card (rw)}
	\minor{8}{/dev/rrom0}{First ROM card (ro)}
	\minordots
	\minor{15}{/dev/rrom0}{Eighth ROM card (ro)}
	\minor{16}{/dev/flash0}{First flash memory card (rw)}
	\minordots
	\minor{23}{/dev/flash7}{Eighth flash memory card (rw)}
	\minor{24}{/dev/rflash0}{First flash memory card (ro)}
	\minordots
	\minor{31}{/dev/rflash7}{Eighth flash memory card (ro)}
\end{devicelist}

\noindent
The read-write (rw) devices support back-caching written data in RAM,
as well as writing to flash RAM devices.  The read-only devices (ro)
support reading only.

\begin{devicelist}
\major{32}{}{char }{Specialix serial card}
	\minor{0}{/dev/ttyX0}{First Specialix port}
	\minor{1}{/dev/ttyX1}{Second Specialix port}
	\minordots
\\
\major{  }{}{block}{Philips LMS CM-206 CD-ROM}
	\minor{0}{/dev/cm206cd}{Philips LMS CM-206 CD-ROM}
\end{devicelist}

\begin{devicelist}
\major{33}{}{char }{Specialix serial card -- alternate devices}
	\minor{0}{/dev/cux0}{Callout device corresponding to {\file ttyX0}}
	\minor{1}{/dev/cux1}{Callout device corresponding to {\file ttyX1}}
	\minordots
\\
\major{  }{}{block}{Third IDE hard disk/CD-ROM interface}
	\minor{0}{/dev/hde}{Master: whole disk (or CD-ROM)}
	\minor{64}{/dev/hdf}{Slave: whole disk (or CD-ROM)}
\end{devicelist}

\noindent
Partitions are handled the same way as for the first interface (see
major number 3).

\begin{devicelist}
\major{34}{}{char }{Z8530 HDLC driver}
	\minor{0}{/dev/scc0}{First Z8530, first port}
	\minor{1}{/dev/scc1}{First Z8530, second port}
	\minor{2}{/dev/scc2}{Second Z8530, first port}
	\minor{3}{/dev/scc3}{Second Z8530, second port}
	\minordots
\end{devicelist}

\noindent
In a previous version these devices were named {\file /dev/sc1} for
{\file /dev/scc0}, {\file /dev/sc2} for {\file /dev/scc1}, and so on.

\begin{devicelist}
\major{  }{}{block}{Fourth IDE hard disk/CD-ROM interface}
	\minor{0}{/dev/hdg}{Master: whole disk (or CD-ROM)}
	\minor{64}{/dev/hdh}{Slave: whole disk (or CD-ROM)}
\end{devicelist}

\noindent
Partitions are handled the same way as for the first interface (see
major number 3).

\begin{devicelist}
\major{35}{}{char }{tclmidi MIDI driver}
	\minor{0}{/dev/midi0}{First MIDI port, kernel timed}
	\minor{1}{/dev/midi1}{Second MIDI port, kernel timed}
	\minor{2}{/dev/midi2}{Third MIDI port, kernel timed}
	\minor{3}{/dev/midi3}{Fourth MIDI port, kernel timed}
	\minor{64}{/dev/rmidi0}{First MIDI port, untimed}
	\minor{65}{/dev/rmidi1}{Second MIDI port, untimed}
	\minor{66}{/dev/rmidi2}{Third MIDI port, untimed}
	\minor{67}{/dev/rmidi3}{Fourth MIDI port, untimed}
	\minor{128}{/dev/smpte0}{First MIDI port, SMPTE timed}
	\minor{129}{/dev/smpte1}{Second MIDI port, SMPTE timed}
	\minor{130}{/dev/smpte2}{Third MIDI port, SMPTE timed}
	\minor{131}{/dev/smpte3}{Fourth MIDI port, SMPTE timed}
\\
\major{  }{}{block}{Modular RAM disk}
\end{devicelist}

\noindent
This device number is provided for older kernels which did not have
the modular RAM disk in the standard distribution.  See major number
1.  This assignment will be removed when the 2.0 kernel is released.

\begin{devicelist}
\major{36}{}{char }{Netlink support}
	\minor{0}{/dev/route}{Routing, device updates (kernel to user)}
	\minor{1}{/dev/skip}{enSKIP security cache control}
\\
\major{  }{}{block}{MCA ESDI hard disk}
	\minor{0}{/dev/eda}{First ESDI disk whole disk}
	\minor{64}{/dev/edb}{Second ESDI disk whole disk}
	\minordots
\end{devicelist}

\noindent
Partitions are handled the same way as for IDE disks (see major number
3).

\begin{devicelist}
\major{37}{}{char }{IDE tape}
	\minor{0}{/dev/ht0}{First IDE tape}
	\minor{128}{/dev/nht0}{First IDE tape, no rewind-on-close}
\end{devicelist}

\noindent
Currently, only one IDE tape drive is supported.

\begin{devicelist}
\major{  }{}{block}{Zorro II ramdisk}
	\minor{0}{/dev/z2ram}{Zorro II ramdisk}
\end{devicelist}

\begin{devicelist}
\major{38}{}{char }{Myricom PCI Myrinet board}
	\minor{0}{/dev/mlanai0}{First Myrinet board}
	\minor{1}{/dev/mlanai1}{Second Myrinet board}
	\minordots
\end{devicelist}

\noindent
This device is used for board control, status query and ``user level
packet I/O''.  The board is also accessible as a regular {\file eth}
networking device.

\begin{devicelist}
\major{  }{}{block}{Reserved for Linux/AP+}
\end{devicelist}

\begin{devicelist}
\major{39}{}{char }{ML-16P experimental I/O board}
	\minor{0}{/dev/ml16pa-a0}{First card, first analog channel}
	\minor{1}{/dev/ml16pa-a1}{First card, second analog channel}
	\minordots
	\minor{15}{/dev/ml16pa-a15}{First card, 16th analog channel}
	\minor{16}{/dev/ml16pa-d}{First card, digital lines}
	\minor{17}{/dev/ml16pa-c0}{First card, first counter/timer}
	\minor{18}{/dev/ml16pa-c1}{First card, second counter/timer}
	\minor{19}{/dev/ml16pa-c2}{First card, third counter/timer}
	\minor{32}{/dev/ml16pb-a0}{Second card, first analog channel}
	\minor{33}{/dev/ml16pb-a1}{Second card, second analog channel}
	\minordots
	\minor{47}{/dev/ml16pb-a15}{Second card, 16th analog channel}
	\minor{48}{/dev/ml16pb-d}{Second card, digital lines}
	\minor{49}{/dev/ml16pb-c0}{Second card, first counter/timer}
	\minor{50}{/dev/ml16pb-c1}{Second card, second counter/timer}
	\minor{51}{/dev/ml16pb-c2}{Second card, third counter/timer}
	\minordots
\\
\major{  }{}{block}{Reserved for Linux/AP+}
\end{devicelist}

\begin{devicelist}
\major{40}{}{char }{Matrox Meteor frame grabber}
	\minor{0}{/dev/mmetfgrab}{Matrox Meteor frame grabber}
\\
\major{  }{}{block}{Syquest EZ135 removable drive}
	\minor{0}{/dev/eza}{First EZ135 drive whole disk}
\end{devicelist}

\noindent
Partitions are handled the same way as for IDE disks (see major number
3).

\begin{devicelist}
\major{41}{}{char }{Yet Another Micro Monitor}
	\minor{0}{/dev/yamm}{Yet Another Micro Monitor}
\end{devicelist}

\begin{devicelist}
\major{42}{}{}{Demo/sample use}
\end{devicelist}

\noindent
This number is intended for use in sample code, as well as a general
``example'' device number.  It should never be used for a device
driver that is being distributed; either obtain an official number or
use the local/experimental range.  The sudden addition or removal of a
driver with this number should not cause ill effects to the system
(bugs excepted.)

\begin{devicelist}
\major{43}{}{char }{isdn4linux virtual modem}
	\minor{0}{/dev/ttyI0}{First virtual modem}
	\minordots
	\minor{63}{/dev/ttyI63}{64th virtual modem}
\end{devicelist}

\begin{devicelist}
\major{44}{}{char }{isdn4linux virtual modem -- alternate devices}
	\minor{0}{/dev/cui0}{Callout device corresponding to {\file ttyI0}}
	\minordots
	\minor{63}{/dev/cui63}{Callout device corresponding to {\file ttyI63}}
\end{devicelist}

\begin{devicelist}
\major{45}{}{char }{isdn4linux ISDN BRI driver}
	\minor{0}{/dev/isdn0}{First virtual B channel raw data}
	\minordots
	\minor{63}{/dev/isdn63}{64th virtual B channel raw data}
	\minor{64}{/dev/isdnctrl0}{First channel control/debug}
	\minordots
	\minor{127}{/dev/isdnctrl63}{64th channel control/debug}
	\minor{128}{/dev/ippp0}{First SyncPPP device}
	\minordots
	\minor{191}{/dev/ippp63}{64th SyncPPP device}
	\minor{255}{/dev/isdninfo}{ISDN monitor interface}
\end{devicelist}

\begin{devicelist}
\major{46}{}{char }{Comtrol Rocketport serial card}
	\minor{0}{/dev/ttyR0}{First Rocketport port}
	\minor{1}{/dev/ttyR1}{Second Rocketport port}
	\minordots
\end{devicelist}

\begin{devicelist}
\major{47}{}{char }{Comtrol Rocketport serial card -- alternate devices}
	\minor{0}{/dev/cur0}{Callout device corresponding to {\file ttyR0}}
	\minor{1}{/dev/cur1}{Callout device corresponding to {\file ttyR1}}
	\minordots
\end{devicelist}

\begin{devicelist}
\major{48}{}{char }{SDL RISCom serial card}
	\minor{0}{/dev/ttyL0}{First RISCom port}
	\minor{1}{/dev/ttyL1}{Second RISCom port}
	\minordots
\end{devicelist}

\begin{devicelist}
\major{49}{}{char }{SDL RISCom serial card -- alternate devices}
	\minor{0}{/dev/cul0}{Callout device corresponding to {\file ttyL0}}
	\minor{1}{/dev/cul1}{Callout device corresponding to {\file ttyL1}}
	\minordots
\end{devicelist}

\begin{devicelist}
\major{50}{}{char}{Reserved for GLINT}
\end{devicelist}

\begin{devicelist}
\major{51}{}{char }{Baycom radio modem}
	\minor{0}{/dev/bc0}{First Baycom radio modem}
	\minor{1}{/dev/bc1}{Second Baycom radio modem}
	\minordots
\end{devicelist}

\begin{devicelist}
\major{52}{--59}{}{Unallocated}
\end{devicelist}

\begin{devicelist}
\major{60}{--63}{}{Local/experimental use}
\end{devicelist}

\noindent
For devices not assigned official numbers, these ranges should be
used, in order to avoid conflict with future assignments.

\begin{devicelist}
\major{64}{--119}{}{Unallocated}
\end{devicelist}

\begin{devicelist}
\major{120}{--127}{}{Local/experimental use}
\end{devicelist}

\begin{devicelist}
\major{128}{--239}{}{Unallocated}
\end{devicelist}

\begin{devicelist}
\major{240}{--254}{}{Local/experimental use}
\end{devicelist}

\begin{devicelist}
\major{255}{}{}{Reserved}
\end{devicelist}

\section{Additional /dev directory entries}

This section details additional entries that should or may exist in the
{\file /dev} directory.  It is preferred that symbolic links use the
same form (absolute or relative) as is indicated here.  Links are
classified as {\em hard\/} or {\em symbolic\/} depending on the
preferred type of link; if possible, the indicated type of link should
be used.

\subsection{Compulsory links}

These links should exist on all systems:

\begin{nodelist}
\link{/dev/fd}{/proc/self/fd}{symbolic}{File descriptors}
\link{/dev/stdin}{fd/0}{symbolic}{Standard input file descriptor}
\link{/dev/stdout}{fd/1}{symbolic}{Standard output file descriptor}
\link{/dev/stderr}{fd/2}{symbolic}{Standard error file descriptor}
\link{/dev/nfsd}{socksys}{symbolic}{Required by iBCS-2}
\link{/dev/X0R}{null}{symbolic}{Required by iBCS-2}
\end{nodelist}

\noindent
Note: The last device is: letter {\tt X}-digit {\tt 0}-letter {\tt R}.

\subsection{Recommended links}

It is recommended that these links exist on all systems:

\begin{nodelist}
\link{/dev/core}{/proc/kcore}{symbolic}{Backward compatibility}
\link{/dev/ramdisk}{ram0}{symbolic}{Backward compatibility}
\link{/dev/scd?}{sr?}{hard}{Alternate name for CD-ROMs}
%\link{/dev/fd?H*}{fd?D*}{hard}{Compatible floppy formats}
%\link{/dev/fd?E*}{fd?D*}{hard}{Compatible floppy formats}
%\link{/dev/fd?E*}{fd?H*}{hard}{Compatible floppy formats}
\end{nodelist}

\subsection{Locally defined links}

The following links may be established locally to conform to the
configuration of the system.  This is merely a tabulation of existing
practice, and does not constitute a recommendation.  However, if they
exist, they should have the following uses.

\begin{nodelist}
\vlink{/dev/mouse}{mouse port}{symbolic}{Current mouse device}
\vlink{/dev/tape}{tape device}{symbolic}{Current tape device}
\vlink{/dev/cdrom}{CD-ROM device}{symbolic}{Current CD-ROM device}
\vlink{/dev/cdwriter}{CD-writer}{symbolic}{Current CD-writer device}
\vlink{/dev/scanner}{scanner device}{symbolic}{Current scanner device}
\vlink{/dev/modem}{modem port}{symbolic}{Current dialout device}
\vlink{/dev/root}{root device}{symbolic}{Current root filesystem}
\vlink{/dev/swap}{swap device}{symbolic}{Current swap device}
\end{nodelist}

\noindent
{\file /dev/modem} should not be used for a modem which supports
dialin as well as dialout, as it tends to cause lock file problems.
If it exists, {\file /dev/modem} should point to the appropriate
dialout (alternate) device.

For SCSI devices, {\file /dev/tape} and {\file /dev/cdrom} should
point to the ``cooked'' devices ({\file /dev/st*} and {\file
/dev/sr*}, respectively), whereas {\file /dev/cdwriter} and {\file
/dev/scanner} should point to the appropriate generic SCSI devices
({\file /dev/sg*}.)

{\file /dev/mouse} may point to a primary serial TTY device, a
hardware mouse device, or a socket for a mouse driver program
(e.g. {\file /dev/gpmdata}.)

\subsection{Sockets and pipes}

Non-transient sockets or named pipes may exist in {\file /dev}.
Common entries are:

\begin{nodelist}
\node{/dev/printer}{socket}{{\file lpd} local socket}
\node{/dev/log}{socket}{{\file syslog} local socket}
\node{/dev/gpmdata}{socket}{{\file gpm} mouse multiplexer}
\end{nodelist}

\end{document}

